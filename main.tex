\documentclass[10pt,landscape]{article}
\usepackage{multicol}
\usepackage{calc}
\usepackage{ifthen}
\usepackage[landscape]{geometry}
\usepackage{hyperref}
% \usepackage{enumitem}
\usepackage{amsmath}
\usepackage{mdframed}

% hippopotato
\usepackage{./hippopotato/hippopotato}

% lazyeqn - math symbols
\usepackage{./lazyeqn/lazyeqn}

\newcommand{\veps}{\vareps}
\newcommand{\ext}{\ensuremath{\textup{ext}}}
\newcommand{\inte}{\ensuremath{\textup{int}}}
\newcommand{\rhoz}{\tfrac{1}{\rho_0}}

\newcommand{\rtwo}{\UI\hspace{-0.07em}\UI}
\newcommand{\rthree}{\UI\hspace{-0.07em}\UI\hspace{-0.07em}\UI}
\def\Bveps{\Bvarepsilon}

\def\mybox#1{\fbox{%
    \addtolength{\linewidth}{-2\fboxsep}%
    \addtolength{\linewidth}{-2\fboxrule}%
    \begin{minipage}[t]{\linewidth}%
      % \vspace{-0.6em}%
      #1%
      % \vspace{0.5em}%
    \end{minipage}%
  }%
}
\def\myboxeqn#1{\fbox{%
    \addtolength{\linewidth}{-2\fboxsep}%
    \addtolength{\linewidth}{-2\fboxrule}%
    \begin{minipage}[t]{\linewidth}%
      \vspace{-0.6em}%
      #1%
      \vspace{-1em}%
    \end{minipage}%
  }%
}

\usepackage{empheq}

\newcommand*\widefbox[1]{\fbox{#1}}
% \newcommand*\widefbox[1]{\fbox{\hspace{0.5em}#1\hspace{0.5em}}}

% \setlist{nolistsep,leftmargin=*}

% Fonts
%\usepackage[garamond]{mathdesign}
%\usepackage{times}
%\usepackage{pxfonts}%ccfonts,times,mathpazo,mathpple
\usepackage[T1]{fontenc}

% Search directory for image files
\graphicspath{{./img/}}

% To Do:
% \listoffigures \listoftables
% \setcounter{secnumdepth}{0}


% This sets page margins to .5 inch if using letter paper, and to 1cm
% if using A4 paper. (This probably isn't strictly necessary.)
% If using another size paper, use default 1cm margins.
\ifthenelse{\lengthtest { \paperwidth = 11in}}
        { \geometry{top=.5in,left=.5in,right=.5in,bottom=.5in} }
        {\ifthenelse{ \lengthtest{ \paperwidth = 297mm}}
                {\geometry{top=1cm,left=1cm,right=1cm,bottom=1cm} }
                {\geometry{top=1cm,left=1cm,right=1cm,bottom=1cm} }
        }

% Turn off header and footer
\pagestyle{empty}


% Redefine section commands to use less space
\makeatletter
\renewcommand{\section}{\@startsection{section}{1}{0mm}%
                                {-1ex plus -.5ex minus -.2ex}%
                                {0.5ex plus .2ex}%x
                                {\normalfont\large\bfseries}}
\renewcommand{\subsection}{\@startsection{subsection}{2}{0mm}%
                                {-1explus -.5ex minus -.2ex}%
                                {0.5ex plus .2ex}%
                                {\normalfont\normalsize\bfseries}}
\renewcommand{\subsubsection}{\@startsection{subsubsection}{3}{0mm}%
                                {-1ex plus -.5ex minus -.2ex}%
                                {1ex plus .2ex}%
                                {\normalfont\small\bfseries}}
% \renewcommand{\paragraph}{\@startsection{paragraph}{3}{0mm}%
%                                 {-1ex plus -.5ex minus -.2ex}%
%                                 {1ex plus .2ex}%
%                                 {\normalfont\footnotesize\bfseries}}
\renewcommand{\paragraph}{\@startsection{paragraph}{4}{\z@}%
                                    {1.25ex \@plus1ex \@minus.2ex}%
                                    {-1em}%
                                    {\normalfont\footnotesize\bfseries}}
\makeatother

% Define BibTeX command
\def\BibTeX{{\rm B\kern-.05em{\sc i\kern-.025em b}\kern-.08em
    T\kern-.1667em\lower.7ex\hbox{E}\kern-.125emX}}

% Don't print section numbers
\setcounter{secnumdepth}{0}


\setlength{\parindent}{0pt}
\setlength{\parskip}{0pt plus 0.5ex}


% -----------------------------------------------------------------------

\begin{document}

\raggedright
\footnotesize
\begin{multicols*}{3}


% multicol parameters
% These lengths are set only within the two main columns
\setlength{\columnseprule}{0.25pt}
\setlength{\premulticols}{1pt}
\setlength{\postmulticols}{1pt}
\setlength{\multicolsep}{1pt}
\setlength{\columnsep}{2pt}

\begin{center}
  \Large{\textbf{cm--c1--cheatsheet}} \\[0.5ex]
  \footnotesize{Computational Mechanics as taught by Lehrstuhl 1 of Institut für Mechanik
    (Bauwesen) in Universit\"at Stuttgart.\\
    Last updated: \today}
\end{center}
{\scriptsize
Note: This material is divided into 2 main parts as mechanics in 1d and 3d.
These represent two separate courses: \emph{C2: Computational Mechanics of
  Materials} --- and --- \emph{E2-02: Micromechanics of Materials and Homogenization
  Methods} as a part of the Computational Mechanics of Materials and Structures
(COMMAS) international MSc. programme.
}

\section{Thermodynamics of 1d Continuum}

\section{Requirements to Material Equations}

\section{Representative Material Models}

\section{Numerical Solution Techniques}

\section{Viscoelasticity}

\section{Elastoplasticity}

\section{Perzyna-type Viscoplasticity}


\rule{0.3\linewidth}{0.25pt}
End of semester 1
\newpage
\section{Thermodynamics of General Internal Variable Formulation}

\subsection{Basic Kinematics at Small Strains}
$\CB \subset \IR^3$ is the continuous body and
$\Vu:\CB\times\UI\rightarrow\CB^3$ is the displacement map at a material point
$\Vx\in\CB$ at time $t\in\UI$.
\begin{align}
  \Vv(\Vx, t) &= \ddelt\Vu(\Vx,t) = \dot{\Vu}\\
  \Va(\Vx, t) &= \ddelt\Vv(\Vx,t) = \ddot{\Vu}
\end{align}
The linear strain tensor:
\begin{equation}
  \Bveps(\Vx, t) = \sym(\nabla\Vu) = \half(\nabla\Vu+\nabla\Vu\tra)
\end{equation}
Decomposed to volumetric--deviatoric ($e$--$\Bveps'$) parts as
\begin{align}
  e &:= \tr(\Bveps) = \Bveps_{11} + \Bveps_{22} + \Bveps_{33} = \Bveps_{\textup{vol}}\\
  \Bveps' &:= \Bveps - \third e\,\Vone = \Bveps_{\textup{dev}}
\end{align}
\subsection{Definition of Stress and Heat Flux}
\begin{itemize}
\item traction vector $\Vt = \widehat{\Vt}(\Vx, t;\Vn) = \Bsigma\Vn$
\item heat flux $\Vh = \widehat{\Vh}(\Vx, t;\Vn) = \Vq\cdot\Vn$
\end{itemize}
\subsection{Balance Equations}
Associated with part $\CP\subset\CB$ are the following physical quantities.

\begin{center}
  \begin{tabular}{r p{0em}<{} l}
    Mass & $m$ & $= \int_\CP\rbr{\rho}\dV$\\[0.8ex]
    Momentum & $\VI$ & $= \int_\CP\rbr{\rho\Vv}\dV$\\[0.8ex]
    M.of Momentum &  $\VL$ & $= \int_\CP\rbr{\Vx\times\rho\Vv}\dV$ \\[0.8ex]
    Kinetic energy & $K$ & $= \int_\CP\rbr{\half\rho\Vv\cdot\Vv}\dV$ \\[0.8ex]
    Entropy & $H$ & $=\int_\CP\rbr{\rho\eta}\dV$ \\[0.8ex]
    Internal energy & $E$ & $=\int_\CP\rbr{\rho e}\dV$ \\[0.8ex]
    Entropy production & $\varGamma$ & $=\int_\CP\rbr{\rho\gamma}\dV$ \\[0.8ex]
  \end{tabular}
\end{center}
with the following quantities
\begin{center}
  \begin{tabular}{l l}
    $\rho(\Vx,t)$ & Density field\\
    $e(\Vx,t)$ & Internal energy field\\
    $\eta(\Vx,t)$ & Specific entropy\\
    $\gamma(\Vx,t)$ & Specific entropy production\\
  \end{tabular}
\end{center}
Global loadings on $\CP$ are
%
\begin{center}
  \begin{tabular}{r p{0em}<{} l}
    Mechanical force & $\VF$ & $= \int_\CP\rbr{\rho\Vb}\dV+\int_{\partial\CP}\rbr{\Vt}\dA$\\[0.8ex]
    Mechanical couple & $\VM$ & $= \int_\CP\rbr{\Vx\times\rho\Vb}\dV+\int_{\partial\CP}\rbr{\Vx\times\Vt}\dA$\\[0.8ex]
    Mechanical power &  $P$ & $= \int_\CP\rbr{\rho\Vb\cdot\Vv}\dV + \int_{\partial\CP}\rbr{\Vt\cdot\Vv}\dA$ \\[0.8ex]
    Thermal power & $Q$ & $= \int_\CP\rbr{\rho\gamma}\dV - \int_{\partial\CP}\rbr{h}\dA$ \\[0.8ex]
    Entropy power & $S$ & $=\int_\CP\,(\frac{\rho\gamma}{\theta})\,\dV - \int_{\partial\CP}\,(\textstyle\frac{h}{\theta})\,\dA$ \\[0.8ex]
  \end{tabular}
\end{center}
Local balance equations are obtained by

\begin{enumerate}
\item Insert the Cauchy-type equations
  \begin{equation}
    \Vt = \Bsigma\Vn \quad , \quad h = \Vq \cdot\Vn
  \end{equation}
\item Apply Gauss-type theorem to convert volume int. to surface int.
  \begin{equation}
    \int_{\partial\CP}\rbr{\cdot}\Vn\dA=\int_\CP\div\rbr{\cdot}\dV
  \end{equation}
\item Apply localization theorem by shrinking the finite part $\CP$ to an
  infinitesimal point $\dV$
\end{enumerate}

\underline{Local forms of the balance equations}

\begin{center}
  \scriptsize
  \begin{tabular}{r l}
    Mass & $\dot{\rho} = 0$ \\
    Momentum & $\rho\dot{\Vv} = \div\Bsigma+\rho\Vb$ \\
    Angular momentum& $\Bsigma=\Bsigma\tra$\\
    Internal energy (1\textsuperscript{st}Law) & $\rho\dot{e}=\Bsigma:\dot{\Bveps}+\rho r - \div\Vq$\\
    Entropy (2\textsuperscript{nd}Law) & $\rho\gamma=\rho\dot{\eta}-\frac{1}{\theta}\rbr{\rho
                                         r-\div\Vq} - \frac{1}{\theta^2}\Vq\cdot\nabla\theta\geq0$\\
  \end{tabular}
\end{center}


\subsection{Material Equations of Continuum Mechanics}
\subsubsection{Thermodynamic Consistency}
Multiply entropy inequality with $\theta$ and substitute $\sbr{\rho r -
  \div\Vq}$ from internal energy balance.
\begin{equation}
  \rho\CD := \rho\theta\eta =
  \boxed{\Bsigma:\dot{\Bveps}-\rho\dot{e}+\rho\theta\dot{\eta}-\frac{1}{\theta}\Vq\cdot\nabla\theta}
  \geq 0
\end{equation}
Split into Clausius--Planck and Fourier Inequalities (CPI) and (FI).
\begin{align}
  \nonumber \rho\CD_{\textup{loc}} & := \Bsigma:\dot{\Bveps}-\rho\dot{e}+\rho\theta\dot{\eta}
                                     \geq 0 \quad \textup{(CPI)}\\
  \rho\CD_{\textup{con}} & := -\frac{1}{\theta}\Vq\cdot\nabla\theta \geq 0 \quad \textup{(FI)}
\end{align}
%
Obtain free energy through Legendre transformation $\boxed{\psi:=e -
  \theta\eta}$ ,
$\dot{e} = \dot{\psi} + \theta\dot{\eta} + \dot{\theta}\eta$.
%
% \begin{align}
\begin{empheq}[box=\widefbox]{align}
  \nonumber \rho\CD_{\textup{loc}} & := \Bsigma:\dot{\Bveps}-\rho\eta\dot{\theta}-\dot{\psi}
                                     \geq 0 \quad \textup{(CPI)}\\
  \rho\CD_{\textup{con}} & := -\frac{1}{\theta}\Vq\cdot\nabla\theta \geq 0 \quad \textup{(FI)}
\end{empheq}
% \end{align}

\subsubsection{General Internal Variable Formulations}
Typical examples of internal variable fields:
viscous strain $\Bveps_v\in\sym(3)\in\IR^6$,
plastic strain $\Bveps_p\in\sym(3)\in\IR^6$,
isotropic hardening variable $\alpha\in\IR$ etc.

A generalized vector of internal variables $\CBI\in\IR^n$ is defined, e.g. for
elastoplastic material with isotropic hardening
$\CBI:=\cbr{\Bveps^p,\alpha}\in\IR^{6+1}$. The internal force dual to
$\CBI\in\IR^n$ is defined as $\CBF\in\IR^n$ .

Canonical dissipative materials are based on two constitutive functions

\begin{itemize}
\item energy storage function $\psi:=\widehat{\psi}(\Bveps,\theta,\CBI)$
\item dissipation potential function $\phi:=\widehat{\phi}(\CBF)$
\end{itemize}

\subsubsection{Coleman's Exploitation Method}
Insert $\dot{\psi}=\partial_{\Bveps}\psi:\dot{\Bveps}
+ \partial_\theta\psi\dot{\theta} + \partial_\CBI\psi\cdot\dot{\CBI}$
into (CPI):

\begin{equation}
\rho\CD_{\textup{loc}} := [\Bsigma-\rho\partial_{\Bveps}\psi]:\dot{\Bveps}
-\rho[\eta+\partial_\theta\psi]\cdot\dot{\theta}
-\rho\partial_\CBI\psi\cdot\dot{\CBI}
\geq 0
\end{equation}
Result:
\begin{empheq}[box=\widefbox]{align}
\nonumber  \Bsigma &:= \rho\partial_{\Bveps}(\Bveps,\theta,\CBI) \\
  \eta &:= -\partial_\theta\psi(\Bveps,\theta,\CBI)
\end{empheq}
Reduced dissipation inequality

\begin{equation}
\rho\CD_{\textup{loc}} := \CBF\cdot\dot{\CBI} \geq 0 \label{eq:cpired}
\end{equation}
where
\begin{equation}
  \boxed{\CBF := - \rho\partial_\CBI\psi(\Bveps,\theta,\CBI)}
\end{equation}
(\ref{eq:cpired}) says that the forces $\CBF$ drive the evolution $\dot{\CBI}$
of internal variables, i.e. $\dot{\CBI}=f(\CBF)$. Written in terms of the
dissipative potential function
\begin{equation}
  \boxed{\dot{\CBI} = \partial_\CBF\phi(\CBF)}
\end{equation}
which requires the definition of initial conditions
\begin{equation}
  \CBI(\Vx,t=t_0) := \CBI_0
\end{equation}
Reduced local dissipation becomes
\begin{equation}
  \rho\CD_\textup{loc} = \CBF\cdot\partial_\CBF\phi(\CBF)\geq 0
\end{equation}



% \subsection{IBVP for Thermodynamic Deformation of Inelastic Solids}


\section{Linear and Nonlinear Elasticity}

\subsection{Basic Constitutive Equations}
No internal variables:

\begin{equation}
  \psi = \widehat{\psi}(\Bveps) \quad , \quad  \Bsigma=\partial_{\Bveps}\psi
\end{equation}

Stress rate:
\begin{equation}
  \dot{\Bsigma} = \BBC:\dot{\Bveps} \quad , \quad
  \dot{\sigma}_{ij}=C_{ijkl}\,\dot{\vareps}_{kl}
\end{equation}
where the 4\textsuperscript{th} order elasticity tensor is defined as
\begin{equation}
  \BBC := \partial_{\Bveps}\Bsigma = \partial^2_{\Bveps{}\Bveps{}}\psi
\end{equation}
For linear elasticity, $\BBC$ is constant, whereas for nonlinear elasticity, it
is a function of strain $\BBC=\widehat{\BBC}(\Bveps)$

\subsection{Isotropic Elasticity}

\subsubsection{Representation of the Free Energy}

\begin{equation}
  \psi = \widehat{\psi}(\VQ\Bveps\VQ\tra)=\widehat{\psi}(\Bveps) \quad \forall\VQ\in\sothree
\end{equation}

where $\VQ$ is a proper rotation in \sothree
\begin{equation}
  \sothree = \cbr{\VQ \,\evat\, \VQ\tra\VQ=\Vone,\, \det\VQ=+1} \label{eq:elas1}
\end{equation}

Spectral decomposition of the strain:
\begin{equation}
  \Bveps\Vn_i = \eps\,\Vn_i \OR \sbr{\Bveps-\eps\Vone}\,\Vn_i=\Vzero
\end{equation}
Principal strains $\cbr{\eps_i}$ and principal directions $\cbr{\Vn_i}$ can be
obtained by the solution of the characteristic equation
\begin{equation}
  \det\rbr{\Bveps-\eps_i\Vone}=\eps_i^3-\Rone_\veps\eps_i^2+\Rtwo_\veps\eps_i-\Rthree_\veps
\end{equation}
Spectral form representation of the strain
\begin{equation}
  \Bveps =\textstyle \sum\limits_{i=1}^3 \eps_i\,\,\Vn_i\otimes\Vn_i
\end{equation}

The invariants $\Rone_\veps$, $\Rtwo_\veps$, $\Rthree_\veps$ read
\begin{equation}
  \boxed{\Rone_\veps=\tr\Bveps}\,;\quad
  \boxed{\Rtwo_\veps=\half\cbr{\Rone_\veps^2-\tr(\Bveps^2)}}\,;\quad
  \boxed{\Rthree_\veps = \det\Bveps}
\end{equation}
%
Using (\ref{eq:elas1}), this yields
%
\begin{equation}
\textstyle  \widehat{\psi}\rbr{\sum\limits_{i=1}^3 \eps_i\,(\VQ\Vn_i)\otimes(\VQ\Vn_i)}
  = \widehat{\psi}\rbr{\sum\limits_{i=1}^3 \eps_i\,\,\Vn_i\otimes\Vn_i}
\end{equation}
%
$\widehat{\psi}$ cannot depend on the eigenvectors (principal directions)
$\cbr{\Vn_i}$ which are arbitrarily rotated by $\VQ\in\sothree$.
Principal strain representation for isotropic elasticity:
%
\begin{equation}
  \widehat{\psi}_\textup{iso}(\Bveps) = \bar{\psi}(\eps_1,\eps_2,\eps_3)
\end{equation}
Since $\cbr{\eps_i}$ are determined by three invariants $\Rone_\veps=I_1$,
$\Rtwo_\veps=I_2$, $\Rthree_\veps=I_3$, one can
alternatively define invariant representation
%
\begin{equation}
  \widehat{\psi}_\textup{iso}(\Bveps) = \widetilde{\psi}(I_1,I_2,I_3)
\end{equation}
State of isotropic elasticity is a reduced space $\IR^3$ (principal strain
space) instead of $\IR^6$ of symmetric tensors.

\subsubsection{Representation in the Principal Stress Space}
\begin{equation}
  \Bsigma = \epartd{\bar{\psi}}{\Bveps}=
  \sum\limits_{i=1}^{3}
  \underbrace{\partd{\bar{\psi}}{\eps_i}}_\text{$\bar{\sigma}_i$}
  \underbrace{\partd{\eps_i}{\Bveps}}_\text{$\Vn_i\otimes\Vn_i$}
\end{equation}
yielding the spectral representation of stress
\begin{equation}
  \Bsigma=\suml{i=1}{3}\bar{\sigma}_i\,\Vn_i\dyc\Vn_i \WITH \bar{\sigma}_i:=\bar{\psi}_{,i}
\end{equation}
Tangent modulus, using chain rule:
%
\begin{align}
 \nonumber \BBC &= \textstyle  \partial^2_{\Bveps\Bveps}\bar{\psi} = \partial_{\Bveps}\Bsigma
  = \partial_{\Bveps{}} \rbr{\suml{i=1}{3}\bar{\sigma}_i\,\Vn_i\dyc\Vn_i} \\
  \nonumber  & \textstyle = \suml{i=1}{3}\suml{j=1}{3} \bar{C}_{ij}\,\Vn_i\dyc\Vn_i\dyc\Vn_j\dyc\Vn_j \\
             & \textstyle + \suml{i=1}{3}\suml{j\neq i}{3}\half
               \frac{\bar{\sigma}_i-\bar{\sigma}_j}{\eps_i-\eps_j}
               \Vn_i\dyc\Vn_j\dyc\rbr{\Vn_i\dyc\Vn_j+\Vn_j\dyc\Vn_i}
\end{align}
where $\bar{C}_{ij}:=\bar{\psi}_{,ij}$.

\subsubsection{Representation in the Invariant Space}
\begin{equation}
  \Bsigma = \epartd{\widetilde{\psi}}{\Bveps}=
  \suml{i=1}{3}
  \underbrace{\partd{\widetilde{\psi}}{I_i}}_\text{$\widetilde{\sigma}_i$}
  \underbrace{\partd{I_i}{\Bveps}}_\text{$\Vb_i$}
\end{equation}
where
\begin{align}
  \nonumber \Vb_1 &= \epartd{I_1}{\Bveps} = \Vone\\
  \nonumber \Vb_2 &= \epartd{I_2}{\Bveps} = I_1\Vone-\Bveps\\
   \Vb_3 &= \epartd{I_3}{\Bveps} =I_3\Bveps^{-1}
\end{align}

\begin{align}
 \nonumber \BBC &= \textstyle  \partial^2_{\Bveps\Bveps}\widetilde{\psi} = \partial_{\Bveps}\Bsigma
  = \partial_{\Bveps{}} \rbr{\suml{i=1}{3}\widetilde{\sigma}_i\,\Vb_i} \\
  \nonumber  &  =
               \suml{i=1}{3}\suml{j=1}{3} \partd{^2\widetilde{\psi}}
               {I_i\partial I_j} \Vb_i\dyc\Vb_j
               + \suml{i=1}{3}
               \partd{\widetilde{\psi}}{I_i}
               \underbrace{\epartd{\Vb_i}{\veps}}_\text{$\BBB_i$}
\end{align}
Here
\begin{align}
  \nonumber \BBB_1 &= \epartd{\Vb_1}{\Bveps} = \IIzero\\
  \nonumber \BBB_2 &= \epartd{\Vb_2}{\Bveps} = \Vone\dyc\Vone - \BBI\\
   \BBB_3 &= \epartd{\Vb_3}{\Bveps} = I_3 \Bveps^{-1}\dyc\Bveps^{-1}+I_3\BBI_{\Bveps^{-1}}
\end{align}
with the following rank 4 identity tensors
\begin{align}
  \nonumber \BBI = I_{ijkl} &= \half(\delta_{ik}\delta_{jl}+\delta_{il}\delta_{jk}) \\
  \BBI_{\Bveps^{-1}} = I_{\Bveps^{-1},ijkl}&= \half (\Bveps^{-1}_{ik}\Bveps^{-1}_{jl}+\Bveps^{-1}_{il}\Bveps^{-1}_{jk})
\end{align}

\subsubsection{Model Problem: Linear Isotropic Elasticity}

\subsection{Decoupled Volumetric--Isochoric Elasticity}

\subsection{Anisotropic Elasticity}

\subsubsection{Representation in terms of Structured Tensors}

\subsubsection{Coordinate free Anisotropy}

\subsection{Finite Element Formulation} % exercise 1

\section{Linear and Nonlinear Viscoelasticity}

\subsection{General Internal Variable Formulation}

\subsubsection{Reduced Representation}

\subsubsection{Convolution Representation}

\subsection{3d Formulation}
\subsubsection{Volumetric Elasticity -- Isochoric Viscoelasticity}
\subsubsection{Alternative Representation}

\subsection{Stress \& Tangent Modulus Updates}
\subsubsection{Algorithm for Rep.B}
\subsubsection{Algorithm for Convolution Rep.}


\section{Rate Independent Plasticity}

\subsection{Ideal Plasticity}

\subsubsection{Elastic Stress Response}

\subsubsection{Dissipative Plastic Stress Response}

\subsubsection{Modulus}

\subsubsection{Model Problem}


\subsubsection{Algorithm: Radial Return Mapping}

\paragraph{Stress}

\paragraph{Tangent Modulus}

\section{Rate Independent Plasticity with Hardening}

\subsubsection{Combined Isotropic and Kinematic Hardening}

\subsubsection{Algorithm}

\paragraph{Stress}

\paragraph{Tangent Modulus}

\section{Rate Dependent Plasticity}

\subsection{Perzyna-type Plasticity for von-Mises Model}

\subsection{Algorithm}

\rule{0.3\linewidth}{0.25pt}\\
\scriptsize
% Copyright \copyright\ 2014 Onur Solmaz\\
\href{http://github.com/nrs/cm-c1-cheatsheet}{http://github.com/nrs/cm-c1-cheatsheet}


\end{multicols*}
\end{document}
